\documentclass[11pt]{article}
\usepackage{amsmath,amssymb,amsthm}
\usepackage{algorithm}
\usepackage{algpseudocode}
\usepackage{framed}
\usepackage{graphicx}

\DeclareMathOperator*{\E}{\mathbb{E}}
\let\Pr\relax
\DeclareMathOperator*{\Pr}{\mathbb{P}}
\DeclareMathOperator*{\var}{\mathbf{Var}}

\newcommand{\eps}{\epsilon}
\newcommand{\inprod}[1]{\left\langle #1 \right\rangle}
\newcommand{\R}{\mathbb{R}}
\newcommand{\F}{\mathbb{F}}

%complexity theory variables.
\newcommand{\Pt}{\ensuremath{\mathbf{P}}}
\newcommand{\NP}{\ensuremath{\mathbf{NP}}}
\newcommand{\PP}{\ensuremath{\mathbf{PP}}}
\newcommand{\BPP}{\ensuremath{\mathbf{BPP}}}
\newcommand{\ZPP}{\ensuremath{\mathbf{Z_{PP}}}}
\newcommand{\RP}{\ensuremath{\mathbf{R_{P}}}}
\newcommand{\coRP}{\ensuremath{\mathbf{coR_{P}}}}
\newcommand{\Ppoly}{\ensuremath{\mathbf{P_{poly}}}}

\newcommand{\Alg}{\ensuremath{\mathcal{A}}}

\mathchardef\hyp="2D

\newcommand{\handout}[5]{
  \noindent
  \begin{center}
  \framebox{
    \vbox{
      \hbox to 5.78in { {\bf CS 388R: Randomized Algorithms } \hfill #2 }
      \vspace{4mm}
      \hbox to 5.78in { {\Large \hfill #5  \hfill} }
      \vspace{2mm}
      \hbox to 5.78in { {\em #3 \hfill #4} }
    }
  }
  \end{center}
  \vspace*{4mm}
}

\newcommand{\lecture}[4]{\handout{#1}{#2}{#3}{Scribe: #4}{Lecture #1}}

\newtheorem{theorem}{Theorem}
\newtheorem{corollary}[theorem]{Corollary}
\newtheorem{lemma}[theorem]{Lemma}
\newtheorem{observation}[theorem]{Observation}
\newtheorem{proposition}[theorem]{Proposition}
\newtheorem{definition}[theorem]{Definition}
\newtheorem{claim}[theorem]{Claim}
\newtheorem{fact}[theorem]{Fact}
\newtheorem{assumption}[theorem]{Assumption}

% 1-inch margins, from fullpage.sty by H.Partl, Version 2, Dec. 15, 1988.
\topmargin 0pt
\advance \topmargin by -\headheight
\advance \topmargin by -\headsep
\textheight 8.9in
\oddsidemargin 0pt
\evensidemargin \oddsidemargin
\marginparwidth 0.5in
\textwidth 6.5in

\parindent 0in
\parskip 1.5ex

%for wrapping inline math mode.
\let\ab\allowbreak

\begin{document}

\lecture{?? --- 23 November 2015}{Fall 2015}{Prof.\ Eric Price}{Sid Kapur, Neil Vyas}

\section{Overview}

In the last lecture we studied graph sparsifiers.
In this lecture we will study Network Coding and Edge Connectivity.

\section{Edge Connectivity}

Define the $s\hyp t$ edge-connectivity to be the number of disjoint paths from $s$ to $t$. Note that this
is the same as the size of the $s\hyp t$ min-cut, which we will denote as $C_{s,t}$. Today, we will study
$s\hyp t$ edge-connectivity on DAGs, or Directed Acyclic Graphs.

Naively, we can use the Ford-Fulkerson algorithm, where we greedily choose paths at each step, flipping
those paths that we used to move from $s$ to $t$. This achieves $O(mn)$ time, where $m$ is the path length
from $s$ to $t$, and $n$ is the flow (???).

Suppose we are interested in the following two variants of this problem: all-pairs and single-source
edge-connectivity. Then, using the algorithm described above, we can achieve $O(mdn^2)$ and $O(mdn)$
time, respectively, where $m,n$ are as above and $d \geq \text{max flow}$; note that $d = \text{max
degree}$ will work.

Can we improve? We claim that we can achieve single-source edge-connectivity in $O(md^2)$. (Due to
Cheung, Lau, and Leung, 2011). In fact, we can achieve acyclic, single-source edge-connectivity in
$O(md^{\omega - 1})$ and all-pairs (not necessarily acyclic) in $O(m^\omega)$, where $\omega$ is the
time complexity of matrix multiplication; we upper bound this by 3 to make the first claim.

The idea behind this analysis is to relate the edge-connectivity problem to network coding.

\section{Network Coding}

\subsection{Introduction}
In each time step, each node can send one ``packet'' of information, and each edge is capable of transmitting
one ``packet.'' Say that we have the following graph, and $s_1, s_2$ have messages $A, B$
respectively, which they want to send to $t_1$ and $t_2$. Our objective will be to saturate the edges of the min-cut, since those
are our ``bottlenecks.'' We can be clever by sending $A$ along the top path, $B$ along the bottom path,
and $A \oplus B$ along the middle path, and decoding $B, A$ at $t_1, t_2$, respectively, using $A \oplus B$.

\begin{center}
\includegraphics[width=0.4\linewidth]{network_coding.pdf}
\end{center}

Suppose that we now want to send a message from $s_1$ to $t_1$, and another from $s_2$ to $t_2$. Then
we can send $\frac{2}{3}$ of each source's message to the non-shared branch and $\frac{1}{3}$ to the
shared branch, achieving $\frac{1}{2}\ell$ bits communicated per round.

In fact, we can send $\geq \ell C_{s,t} (T - n)$ bits in $T$ rounds, where $\ell$ is the number of bits
each node can communicate (size of the message, in our case), and $n$ is the length of the path the
messages travel.

\subsection{Unknown / Changing network structure}

The earlier methods relied on our knowledge of the network's structure to achieve more efficient
communication. However, we might not know the structure of the network, or it might be changing in time.
Thus, we would like a method that is oblivious to the overall network structure. We will employ the idea
of sending components of the messages obliviously and then reconstructing the full message at the destination.
Suppose we have many source nodes, with distinct messages, trying to communicate with one destination $t$.

Suppose that our input is of the form
$$
m_1, \ldots, m_k \in \F_q^r;
$$
note that this input is $kr\log(q)$ bits in size.

To each of these vectors, prepend the vector $e_i \in \F_q^k$, where $i$ is the index of the input;
this makes the input vectors linearly independent and guarantees that we can recover them exactly through
Gram-Schmidt. Call the vectors of this augmented input $\hat{m_1}, \ldots, \hat{m_k} \in \F_q^{r+k}$.
Now, take the subspace $X_s$ defined to be the span of $\hat{m_1}, \ldots, \hat{m_k} \in \F_q^{r+k}$.

Our goal is now to transmit $X_s$ efficiently to $t$. This is a simpler problem as, since we're transmitting a subspace,
we can transmit any $k$ linearly independent vectors in the subspace, and then we can recover the basis
$\hat{m_1}, \ldots, \hat{m_k}$, and from there our original messages.

Suppose that the capacity of each edge is $(k+r)log(q)$ bits, i.e. we can transmit one vector in the
subspace per edge per time step.

% H option ensures that the algorithm is placed "Here",
% i.e. does not float
\begin{algorithm}[H]
\begin{algorithmic}
\Function{SubspaceCommunication}{}
  \ForAll{$(v, v^\prime) \in \text{DAG-ordered edges}$}
    \State $v$ samples $m \in X_v$
    \State $v$ sends $m$ to $v^\prime$
    \State $v^\prime$ ``adds'' $m$ to $X_{v^\prime}$
  \EndFor \\
  \Return $X_t$
\EndFunction
\end{algorithmic}
\end{algorithm}

We claim that there is a $1 - \frac{m}{q}$ chance that $v^\prime$ ``learns'' something from $v$ in this
procedure, given that $X_{v^\prime} \neq X_v$. This is because $\exists u\in X_v, \notin X_{v^\prime}$,
and in order for $v^\prime$ to ``not learn'' anything from $v$, the coefficient of $u$ in $m$, the message
that $v$ sends to $v^\prime$, must be 0. Since we are over the field $\F_q$, this happens with probability
$\frac{1}{q}$.

Note that we only have a log-dependence on $q$ (See the size of the input, $kr\log(q)$, so we can set
$q$ to be a very large prime.

Suppose there are $C$ disjoint paths (there are $C_{s,t}$ such paths). Then $\forall v_1, \ldots, v_C$,
with $v_i$ in the $i^{\text{th}}$ of $C$ disjoint paths, $\exists u_1, \ldots, u_C$, where $u_j \in X_{v_j}$,
and $X_{v_j}$ is the subspace ``by the time'' $v_j$ is reached, such that the $u_i$ are linearly independent
with probability $1 - \frac{m}{q}$.

The proof follows by induction on time.

Suppose $\exists u_1, \ldots, u_C$ with $u_j \in X_{v_j}$, and the $u_i$ are linearly independent.
Send $u$ from $v_i$ to $v_i^\prime$. Then we want to show that $\exists u_1 \in X_{v_1}, \ldots, u_i \in
X_{v_i}$ such that these $u$ are linear independent.

This is true if $u \perp \{u_1, \ldots, u_{i-1}, u_{i+1}, \ldots, u_C$, else this is true if
$\left<u, u_i\right> \neq 0$. Note that this second condition occurs with probability $1 - \frac{1}{q}$.
Since we make at most $m$ moves total, we have that ``they all work,'' i.e. the vectors are linearly
independent in each steap, with probability $1 - \frac{m}{q}$. Note that $m$ here is the length of the
path from $s$ to $t$, not the vectors of the input.

Thus, we have that we need to make $m$ steps, and in each step we perform $d^2log(q)$ work, since it
takes $log(q)$ time to operate on vectors in $\F_q$ and (the $d^2$ is apparently from matrix-vector operations,
or something; I also have in my notes that this is the time it takes to check all those inner products).
So our time complexity for bit operations is $O(md^2log(q)$ with success probability $1 - \frac{1}{m^c}$,
and thus our time complexity for word operations is $O(md^2)$, as desired.

Now we address the method of sampling $u$ from $X_v$. Set $X_v = U_v^TY$, where $Y \sim \F_q^{|X_v|}$ are
i.i.d. uniform. Then let $u^\prime = u - U_vU_{v^\prime}u$. Note that the term being subtracted should
remind you of projection. Then $u^\prime = 0 \iff u \in X_{v^\prime}$, so if $u^\prime = 0$, set
$U_{v^\prime} = \left[\frac{U_v}{u}\right]$.

Note that we have, throughout, relied on some ordering of the vertices. This ordering is DAG ordering,
in which we only transmit from a node if it has received all (possible) incoming messages. For example,
in the following graph, if we don't employ DAG ordering, it's possible that we reach the destination
``too soon,'' and the subspace we've transmitted isn't of high enough dimension; suppose we take path
$(i)$, for example.

\begin{center}
\includegraphics[width=0.4\linewidth]{dag_ordered.pdf}
\end{center}

\subsection{Continuous Transmission}

This algorithm involves every vertex communicating at most once, so what about continuous transmission?
We'll give a cursory look at the gossip algorithm.

After $T$ rounds, $t$ receives $X$ with dimension $\geq C(T-n)$ w.p. $1 - \frac{(T - n)m}{q}$,
where $k = C(T-n)$. This implies that we've sent $(r\log(q))C(T-n)$ bits. We claim that $r\log(q)
CT$ is optimal for any protocol.
\end{document}


